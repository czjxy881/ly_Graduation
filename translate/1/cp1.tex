\ifx\allfiles\undefined
\documentclass{XDBAthesis}
\def\pictures{}
\begin{document}
\else
\fi
\chapter{简介}
要说手势控制最为吸引人的一点,应该就是我们通过手势控制可以直接和目标设备进行交互,而不需要其他多余的设备了。就像我们可以通过手势控制来直接空调电视而无需遥控器。近几年,有很多这方面的研究,在第六章我们大致介绍下合手势控制相关的一些研究成果。手势控制有着广阔的应用前景,像控制个人设备,可视化交互手段,控制机械系统,玩电子游戏等等。

本文的主要目的是演示一种基于形状特征和颜色特征的实时手势追踪和姿势识别系统。我们主要通过两个方法来实现这一系统:(i)在手的表示,定量分类模型中的彩色特征检测,(ii)同步追踪和姿势识别中利用分层采样来进行颗粒过滤。

\chapter{手表示方法}



\ifx\allfiles\undefined
%\bibliographystyle{unsrt}
%\bibliography{main}
\end{document}
\fi