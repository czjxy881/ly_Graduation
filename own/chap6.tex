\ifx\allfiles\undefined
\documentclass{XDBAthesis}
\def\pictures{}
\begin{document}
\else
\fi
\chapter{总结与展望}

基于视觉的手势识别长期以来都是人机交互领域研究的热点。本课题在近两年来智能移动终端设备爆炸式增长的大背景下,针对智能移动终端设备的特点,对手势的检测和识别方法做了深入研究,并选取 Android 智能平台为应用背景,设计了一个 Android 平台上的实时手势交互系统,这是在智能移动设备上研究人机交互手段所做出的新尝试,本文所做的主要工作和取得的成果主要以下有几点:
\begin{itemize}

\item 第一,在OpenCV调用摄像头与处理图像占用CPU方面进行的分频处理,使得在处理时间和效率上得到了大大的提高,并且直线检测的速度较快,计算量较小,和同类处理方式相比适合在移动设备上使用。

\item 第二,在手势识别方面,本文首先研究了手势分割技术,良好的手势分割是识别的前提,本文通过边缘检测和直线提取两个步骤得到了较好的手势分割效果。

\item 第三,在应用方面,本文设计了一个基于 Android 平台的实时手势识别系统,还演示了通过调用该库实现的一个使用特定手势控制摄像头完成拍摄的应用案例,这是研究智能终端设备上人机交互技术的一次新的尝试。
\end{itemize}

目前在识别过程仍有一些不足,归纳起来,造成错误识别的主要原因有以下两条:
\begin{itemize}
\item 第一,手势是3D的图像,实验中采用了2D的形式来表现3D图像,造成了一些手势信息的丢失,这是错误识别的主要原因。

\item 第二,获取的图像质量不高,背景中的景物比较复杂,这些由景物产生的像素对手势轮廓的提取造成了干扰,使最终的特征提取造成错误。

\end{itemize}
本课题在未来还可以在以下四个方面加以研究:
\begin{itemize}
\item 第一,在现有的手势检测基础上进行手势跟踪,本文没有研究手势跟踪的算法,但如果要达到更好的交互效果就应该加入手势跟踪。

\item 第二,降低软件的CPU和内存的占用率,目前完成的图像处理软件在这方面做得并不够好,占用手机的资源比较多。

\item 第三,实现图像亮度自动调整的功能,由于日常生活中由于曝光不足的照片很常见,亮度自动调整具有比较现实的意义。可以增加一些特定功能,比如针对现有的年轻人群对照片的要求,增加美白,瘦脸,瘦身等功能

\item 第四,完善 Android 手势拍照应用,进一步提高软件的稳定性和识别的准确性,使其在实际生活中具有实用性,争取将其应用发布到 Android 应用市场,供广大Android 设备用户使用。
\end{itemize}

\ifx\allfiles\undefined
%\bibliographystyle{unsrt}
\bibliography{main}
\end{document}
\fi