\ifx\allfiles\undefined
\documentclass{XDBAthesis}
\def\pictures{}
\begin{document}
\else
\fi
\chapter{绪论}
\section{课题研究背景及意义}

在科技高速发展、研究层出不穷、需求不断更新的今天,外设硬件或屏幕触碰作为机械式输入设备,在某些方面来说,很难进行3D和高度自由的输入,这种交互方式对人们的日常生活来说并不方便,需要一个学习并且适应的过程。随着研究工作的更加深入,人们更多的开始关注人脸识别、人眼跟踪、人体跟踪与识别及手势跟踪与识别等更加符合人类习惯的人机交互技术。

语言(包括书面语言和口语)是人与人之间交互的重要形式,但人类的肢体形态语言如手势、体势和表情等,也是常用的交互方式。人与计算机或手机的交流方式较呆板,相对而言人与人之间的肢体语言交流更加多种多样。因此,研究人机交互形式,人体语言与有声语言的融合,对于提高机器视觉,进而提高人机交互接口的实用性是非常有意义的。“手势”在人机交互中,不单单具有生动、形象而且直观的特点,更具有很强的视觉效果,因此手势完全可以作为一种交互手段。

随着人机交互技术的发展,最初的交互方式大部分是以实体按键方式控制智能设备,而现在更多的发展为以屏幕触控方式为主,甚至有些设备已经完全淘汰了实体按键的交互方式。然而,随着用户需求的不断提高以及开发者头脑风暴的创意不断增加,不仅仅传统按键被市场渐渐淘汰,目前主流的屏幕触控方式也不能完全满足用户需求,这些操作方式都有一定的局限性,主要体现在用户需要接触到设备才能够完成操作,因此手势识别突破了距离的限制,给用户带来了全新的体验。随着这种以非接触式自然手势操作技术在Android系统中逐渐发展成熟,必将带来一场全新的技术革命,为开发者开拓眼界,更新了思维模式,提升在现在的发展中,很多软件都与手势识别相结合,如红外设备、Kinect等。但是对于摄像头识别手势拍照系统并没有很成型的作品,在拍照过程中可能因为一手持手机,另一只手拿东西或其他原因对按键造成不便,能把 OpenCV 移植到 Android 平台上使用是对图像图像处理在了市场竞争力。

移动终端上方便的进行具有的巨大意义。因此在拍照过程中加入手势识别是很有发展空间与发展潜力的项目之一。

\section{课题研究现状}

\subsection{Android软件发展现状}

Android系统自推出以来,就以明显的优势逐渐扩大自大的市场份额,尤其在国外,其呼声日高,可谓是如日中天,正处于蓬勃发展的开拓阶段。据美国某市场调研机构2012年发布的一份最新报告显示。2012年一季度在美国,基于Android系统的智能手机的销售量已占据全美手机销售量的28\%份额,而大名顶顶的IPhone手机其市场份额紧追其后,占到21\%的市场份额,已经确定了Android系统的市场占有比。据业内人士分析,随着Android系统相应软件的不断开发应用,选择Android系统手机或者无线终端设备的人会越来越多,其市场霸主的地位在更新更好的系统出现之前是不可动摇的。

  未来基于Android系统的应用软件将进入飞速发展的全新阶段。Android系统的应用绝不仅局限于手机产业,几年来其迅速扩张到相关领域,例如平板电脑、车载系统、电视STB、智能电器、智能会议系统等。目前,各IT厂商都在努力的研发前沿应用软件,以期在Android系统发展这一群雄逐鹿的关键阶段,占领更多的市场份额。

目前对Android的发展方向一类是偏向硬件驱动,一类是偏向软件应用。从目前的招聘需求来看,后者的需求最大,包括手机游戏、手机终端应用软件和其他手机应用软件的开发。随着各种移动应用和手机游戏等内容需求日益增加,也将激励大中小型手机应用开发商加大对Android应用的开发力度,因此Android人才的就业前景也非常广泛。几乎每一个android手机用户都是游戏的需求者,都是潜在的顾客,现今的android用户不过是冰山一角,随着android手机市场进一步壮大,游戏的市场容量将具备较大的增长空间,游戏开发者不会愁吃不饱,只会愁胃口不够大。

目前国内的Android开发还是主要以应用开发为主,主要分成3类:为企业开发应用、开发通用应用(放到Android Market或者其他App Market销售)以及游戏开发(放到Android Market或者其他App Market销售)。第一类开发者一般身处规模较大的公司,这些公司主要为自有品牌或者其他品牌设计手机或者平板电脑的总体方案。除了根据需求对系统进行定制外,更多的工作在于为这些系统编写定制的应用。第二类开发者,一般处于创业型公司或者是独立开发者,他们的盈利方式主要是2种:为国外公司进行外包开发,或者通过Google的移动广告(AdMob)通过广告点击分成。而理论上的通过付费下载的形式来盈利的,现在国内鲜见成功者。第三类开发者,目前和第二类开发者类似。

丰富的应用软件程序有游戏、生活、新闻、阅读器、记事本、天气预报、文件或行事历管理及硬件管理方面的服务软件,而工具型软件将随着Android 系统日趋完善而减少,除了 UI 客制化及动作感测等结合硬件及软件的应用外,扩增实境应用也将逐渐进入 Android 应用领域中,更是值得关注的发展方向之一。

\subsection{手势识别研究现状}

手势是人类沟通交流不可分割的一部分:人们在相互交流时总会指手划脚。而手势识别技术又开辟了我们与机器、设备或电脑互动的新局面。手势识别的前景非常令人期待,特别是对于智能手机和平板电脑而言。这些设备已经深入人们生活的各个方面,而新的通信接口始终非常受人们的欢迎。

采用摄像头跟踪进行手势识别的技术实际上已经使用了一段时间。领先的游戏机,如微软的Xbox和索尼公司的Play Station,都配置了手势识别设备(分别是KINECT和Play Station Eye)。目前,这两种设备已经升级至第七代和第八代。不过,移动设备的手势识别技术仍面临几个重要问题,包括在不利的光线条件下,该技术能够实现的效果,背景的变化与高功耗等。然而,业内普遍认为,这些问题可以通过不同的跟踪解决方案和新技术克服。

这两年来智能移动终端设备特别是 Android 设备数量迅猛增长,很多开发者也将目光投降了除了触控以外其它的交互方式,例如语音识别技术,现在已日趋成熟,手势等肢体语言的交互形式将是另一个发展趋势,并有可能成为发展主体。近年来陆续出现了一些使用肢体语言交互的应用,如眨眼拍照的相机应用、笑容触发的相机应用、眼球转动来控制的阅读器应用等。相比之下更直观的手势交互应用必将成为热门。

\section{论文主要内容安排}

本论文共分为五章,内容安排如下:
\begin{itemize}
    \item 第一章是绪论,本章主要介绍了课题研究背景及意义、Android软件发展现状及手势识别的发展现状。
    \item 第二章是算法涉及到的技术谈论,本章对论文中所涉及到的相关关键技术进行了系统的分类与介绍。其中包括Android系统开发平台架构原理、NDK简介及OpenCV简介。
    \item 第三章是基于OpenCV实现Android系统对摄像头的调用及操作。
    \item 第四章是图像处理及特征手势的提取算法。
    \item 第五章是软件实现与实验结果。
\end{itemize}

\ifx\allfiles\undefined
%\bibliographystyle{unsrt}
\bibliography{main}
\end{document}
\fi