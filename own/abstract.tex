\begin{abstract}
    Android操作系统在智能手机、车载系统、平板电脑等领域现在正发挥着越来越大的作用,稳定、开封且免费的Android操作系统也受到了各大厂商的喜爱。然而,如今拥有的人机交互方式仍有一定的局限性,这种局限性主要体现在人机交互的距离上。传统点触式的人机交互方式已不足以满足当今人们的需求,基于视觉的手势识别俨然已成为人机交互中的新研究方向。以手势直接作为输入,人机之间不再需要中间媒介,即不触摸到屏幕也可对手机进行操作。
    
文中首先介绍了 Android 系统和 OpenCV 库的背景及架构原理,论述了当前 Android设备的人机交互方式;详细讨论了移植 OpenCV 库并部署到 Android 系统中的方法,其中 OpenCV 库是作为支撑识别自然手势信息处理或进行其它开发的基础图像处理函数库;重点论述了基于 Android 平台实现自然手势识别系统的整体框架设计,以及组成系统的各个模块的设计与实现方法;详细介绍了当Opencv和照相机同时调用摄像头时的解决方法与实现。
    
最后,对自然手势识别系统进行了测试。此系统可以根据分类器识别用户的手势动作,然后做出相应的决策,测试结果符合实时性检测要求,整体运行达到了预期的效果。这种交互方式更符合用户的自然行为习惯,交互过程中用户只需通过手势来传达命令给Android 智能设备,随着这种非接触式人机交互技术的不断成熟与完善,它将成为Android 智能设备的一种主要操控方式。
\end{abstract}