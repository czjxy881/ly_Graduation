\ifx\allfiles\undefined
\documentclass{XDBAthesis}
\def\pictures{}
\begin{document}
\else
\fi
\chapter{基于OpenCV实现Android系统对摄像头的调用及操作}

\section{OpenCV环境搭建}

在搭建好Android开发坏境的前提下,下载Cygwin,配置NDK。配置好后,下载OpenCV for Android编译,进入Cygwin shell执行(注意路径中的空格):svn checkout http://android-opencv.googlecode.com/svn/trunk/  android-opencv-read-only

    之后进入OpenCV目录执行sh build.sh编译,编译完成后即可以使用Eclipse+ADT的方式完成OpenCV程序的开发。

\section{架构设计}

    在已经搭建好了编译环境后,接下来主要进行两部分的开发:首先主要是涉及 UI 界面和程序的逻辑流程在基于Android应用程序框架下进行Java端的开发;其次是JNI接口的开发,通过OpenCV与JNI接口编写本地的 C/C++ 代码,并利用 Android NDK 对其进行编译,然后运用编译后生成的 Java 代码可调用动态链接库so文件,最后通过 Eclipse 编译打包并生成应用程序,整体框架图如图\ref{fg:whole}所示。
\begin{figure}[htb]
    \centering
    \includegraphics[width=0.8\textwidth]{figure/opencv}
    \caption{整体框架图}
    \label{fg:whole}
\end{figure}


\section{highGUI库函数进行对摄像头的调用及操作}

先建立一个窗口,用来显示图像。通过cvCreateCameraCapture来读取摄像头。该函数的输入参数是一个ID号,只有存在多个摄像头时才起作用。对摄像头成功调用后,可以采取对图像的目标算法来实现对摄像头采集到图像的相应操作。函数cvGrabFrame从摄像头或者文件中抓取帧。被抓取的帧在内部被存储。这个函数的目的是快速的抓取帧,这一点对同时从几个摄像头读取数据的同步是很重要的。

\section{照相机和Opencv同时对摄像头占用的处理方案}

    传统的处理方法是将图像一帧帧传入数据层,数据层调用图像处理函数完成对图像的处理后,再将图像传入表现层,使前端可以显示出当前图像。但是由于CPU有限,处理时间较长,会造成很明显的延时导致用户使用效果不理想。因此针对延时问题,采用多线程处理方案,当摄像头捕捉到图像后,对传入图像进行分频处理,即一帧传入数据层,一帧传入表现层。由于图像捕捉每一帧时间极短,因此在表现层也不会出现断帧的现象。分频处理不单单使画面更加流畅,还使软件在实时性的效果更好,图片处理更加快速。

\section{实现意义}

   在 Android 平台上利用 OpenCV 的 highGUI 库进行对摄像头的调用及操的实现充分说明了移动终端对摄像头图像的采集及后续处理的可行性。Android NDK的灵活开发方式,可以让程序员在 PC 机下快速地做完核心算法的测试,然后再方便地移植到 Android 平台下,将界面设计和算法程序彻底的分割开来,提高了程序的编写效率。随着移动终端设备的发展,移动终端的高速处理和高性能也将越来越普遍,更复杂的应用程序将能够更完美地在移动终端上运行,所以当实现Opencv在Android系统上的调用,在Android系统上的图像处理和计算机视觉这些应用程序也会大大增加。

\ifx\allfiles\undefined
%\bibliographystyle{unsrt}
\bibliography{main}
\end{document}
\fi